\paragraph{}
Criminals and terrorists are increasingly exploiting technological progress to conduct their business, consequently even states and law enforcement agencies must evolve their way of investigating. The use of malware to fight crimes is nothing new but for long time utilization of malware as investigative tool has been kept secret and unregulated. This lack of norms lead to numerous scandals that shown the necessity to regulate this technology. This paper aims to point out some problems that can threat citizen's right in the event that spyware is used inappropriately and without proper rules. It focus on the need to regulate the use of malware by states and the importance to properly monitor the firms that are involved in the manufacturing of the software adopted by the law enforcement. The topic intersects with numerous issues addressed by jurists and lawyers, which are all based on ethical behavioral norms.       
\paragraph{}
The first two sections of the document explore why law enforcement agencies need to use these tools, shows some news cases involving the use of this technology, and briefly explain some features of the malware, necessary to understand the subsequent discussions. The third section focus on the issues that lack of regulation can lead to, showing how much powerful are malware and comparing it with older investigation techniques. After that, the attention is shifted towards the importance of controlling supplier companies so that they do not support espionage activities that violate human rights. 