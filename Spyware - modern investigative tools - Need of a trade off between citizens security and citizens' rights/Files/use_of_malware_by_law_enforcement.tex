\paragraph{}
Before starting to argue and discuss any possible problem related to the utilization of malware in name of national security, is important to understand why governments have necessity to supply law enforcement with this kind of technology. In a wider context,  malware can be classified as cyberweapons, justified by necessity to fight the growing threat of cyber terrorism, perpetrated by terrorist organizations, sometimes sponsored by adversarial states \cite{cyber_attacks_and_terrorism}. For this reason is legitimate to ask if the utilization of these technologies is admissible outside a cyberwar context. In my opinion their utilization can be morally justified considering two principle: the law of international conflict, which regulate and legitimate the use of armed force as countermeasure to a possible threat, that can be extended to conflicts in cyberspace, and the principle of self-defence that allow states and non-states entity to use force in order to protect themselves. These aspects are well presented by D. Denning in "The Ethics of Cyber Conflict"\cite{cyber_conflicts}, and in conclusion, utilization of spyware by law enforcement is permissible for defensive purpose, but response must be carefully proportioned to the threat, and subjected to strict controls and rules.\\     
Anyway, without further explore the cyberwar scenario, the utilization of malware by law enforcement is easily understood analysing how information is exchanged over Internet. "The government’s need to use malware to investigate crime is almost always connected to a target’s successful use of encryption" \cite{malware_justification_1}, "As encryption and anonymization tools become more prevalent, the government will foreseeably increase its resort to malware"\cite{malware_justification_2}. As we can see from the two previous quotes, Paul Ohm, Jonathan Mayer and many other experts \cite{malware_justification_3} agree that the main reason to use malware is the end-to-end encryption of the communication. As the correct implementation of the encryption algorithm makes useless any interception, with the installation of a malware directly on the device of the target allow to eavesdrop the information in plain text. There are many successful utilization of this technology that are jumped on the front page of newspaper, in example the recent European conjuncted operation that lead to dismantling EncroChat, an encrypted network, and to arrest about hundreds of criminals \cite{encrochat_1}\cite{encrochat_2}. However this kind of techniques, and their purpose, are also involved in many scandals such persecutions on journalists, activists or citizens \cite{abuse_1}\cite{abuse_2}\cite{abuse_3}. 
\paragraph{}
It's clear that in some context utilization of malware is necessary and is the only solution against criminals that hide their identity or exchange information using encrypted channels.     