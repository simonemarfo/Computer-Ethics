\paragraph{}
From now on I want to analyze and discuss some troubles and damage that such an invasive and persistent software can lead to, if not used with strict regulations. First I try to point out the main problems regarding the utilization and the users, comparing this investigative tool with other forms of surveillance. After that I want to highlight a problem regarding the developing and the selling of this software, and some possible scenarios where preciosity of the vulnerabilities could harm an innocent person in court.

\subsection*{The need to tame Trojan horses}
\addcontentsline{toc}{subsection}{The need to tame Trojan horses}
\paragraph{}
Governments have been secretly using malware for many years, but management of these tools cannot be left to chance. Fortunately in recent years, thanks to some information leaks that reveal this practice to the citizen, States and organization starts to discuss and to regulate this kind of software, for instance in Italy the decree law No. 216/2017\cite{dl_216/2017} and its subsequent update (d.l. No. 28/2020\cite{dl_28/2020}) aims to this. One of the main aspect to consider is when spyware can be used and against whom. In order to avoid misuse is important to define some special case in which law enforcement can require the utilization of malware, and this kind of situation are restricted to the case in which the interception is strictly necessary for the continuation of the investigation and other techniques are ineffective. The Italian decree law No. 28/2020 define the utilization of spyware in cases where the survey relate to some specific crime \cite{28/2020_ref}. Some other problems are harder to regulate because more related to the information technology field. Is important to ask whether a government can use a malware only in investigations that are done over its territory, or it can deploy an attack against someone that live in another country. Other considerations can be done trying to analyze the four steps necessary to perform a wiretap with a malware: delivery, exploitation, execution and reporting. The delivery phase require to find a way to install the spyware in the victim machine, this can be done through social engineering technique or manually, but can happen that these methods fails. One solution is to exploit some other delivery network, install the spyware on thousand of device and, at the end, collect the data of the interested party only. This is what happened for example with the German spyware case\cite{R2D2} or the Italian scandal of Exodus\cite{exodus}. These situations, where spyware launched 'in the wild', highlight the importance of a solid regulation since is not always possible to discriminate a specific target. Therefore, we can debate if launch a spyware 'in the wild' in name of national security is justifiable, but as Rossano Ferraris wrote "The answer is simple: it is not possible!"\cite{rossano_ferraris}, if the use of malware is justified against criminals, install it in most of citizens'devices, without evidence of crime, cannot be justified in any way, especially considering the abuses that can be done with this tool. I will explain the issue related to the exploitation phase further on because strictly connected to the concept of vulnerability, now focus on the execution phase. I think that there are three important questions to answer in order to try to regulate the use of malware: how long they can stay active, when they can collect data and, more important, what exactly can do. This remote intrusion software, usually, are not provided with a switch off button, moreover is difficult to discover them. These properties imply that it can remain active for a long, and nobody can uninstall it without reveal the presence of it. This may be a problem in the cases in which the investigator have not enough evidence to accuse the investigated, and the survey are abandoned. The second questions regard the moments in which this tools can perform their work. With the old wiretap techniques, investigators had to install hidden microphones, physically stalk the suspects or listen their phone calls. Using malware all this thing can be done better, more easily and with less risks. This, of course, damage the privacy of the investigated, someone can assert that is irrelevant because we are trying to unmask a criminal, and I could be agree, but, also the privacy of every other individual that interact with the suspect is threatened. With the old techniques the amount of information collected were proportional to the efforts of the police and limited to the business they were investigating on, while with the use of malware the data that can be collected are not only related to the criminal activity, but to anything happening near the infected device. To make it more clear, if the spyware is collecting images from the camera and accidentally frame a loving betrayal of two strangers we get into a privacy violation. I will expand the third question in detail later on, for now I want to point out that spyware provide a power that the traditional wiretapping techniques did not have: they are not only passive tools, they are active, can also "use" the device. In the reporting phase the collected data have to be transmitted through the internet and stored in a computer. In the d.l. No. 216/2017 is not defined any requirements concerning where store the information and only some generic indication are given about the way this data have to be transferred to destination server\cite{216/2021_reporting}. The regulation of these aspects is crucial to guarantee the right of privacy of citizen since they are handling sensitive data. Strict requirements have to be provided defining the communication protocols and also imposing that the information have to be stored in an infrastructure managed by the state, avoiding any possibility of outsourcing to some well known tech company, which provide a cloud storage service, but aims firstly to increase its incoming.

\subsection*{Other forms of surveillance}
\addcontentsline{toc}{subsection}{Other forms of surveillance}
\paragraph{}
As I wrote before, is important to allow the use of spyware by law enforcement only when this type of technology is strictly necessary for the continuation of the investigation. In the other cases exist other techniques that provide the same results. I already spent some words comparing spyware and old investigative techniques, now I want to focus on which are the differences between Trojan horses and other two technology used by law enforcement: cell-site simulators and facial recognition. Cell-site simulator, or IMSI catchers, are devices, controlled by police, that emulate legitimate telecommunication  infrastructure and trick phones into connecting to these devices rather than a cell-phone tower. These devices are used to better triangulate the location of a phone, intercept the conversation that pass through it and, sometimes, can alter the content of communications. Seems that IMSI catcher are almost invasive like spyware, or worst, because also innocent users are duped by the fake cell-site. However is hard that the wiretap with this technology last for long time, because is likely that the suspect change its position during the day and is expensive for law enforcement to stalk him. Moreover I want to point out something that I had not explored before, the fact that exploiting a vulnerability, i.e. use a malware, can be morally questionable because broke an unwritten contract of trust between the user and the service provider, that guarantees to the buyer the product is secure. Instead, as Paul Ohm wrote \cite{cell_site_simulator}, cell-site simulator take advantage of the openness of the cell phone standards and the necessity of backwards compatibility, meaning that no bugs or system misconfigurations are exploited. Regarding the face recognition technology, even if can be more easily abused by law enforcement, due to the fact that image comparison algorithm can be used multiple times, without wasting any resource and with a low deployment of forces, this technology is limited to take in input photos or video of an anonymous people, that can come from social networks or other sources, and provide as output the (likely) identity of the suspect. All three technology can be misused, for this reason all have to be regulated, but, neither the cell-site simulator nor the facial recognition, allow to collect the same amount of information as with a spyware. Malware are not only a wiretapping tools, they are persistent and, more important, can act, having full access to the device.

\subsection*{Malware manufactures and moral issues}
\addcontentsline{toc}{subsection}{Malware manufactures and moral issues}
\paragraph{}
I already provided some example of abuse of this technology, all of them perpetrated by liberal states, and it's clear that, if such abuse can happen in liberal states, in dictatorial states, where humans rights violations are commonplace, this type of technology can only increase to the detriment of citizens' freedom. In 2015
Hacking Team was an Italian information technology company in the offensive security field, able to develop powerful intrusion tools which could be used by anyone after a two week course. Their spearhead was the Remote Control System called Galileo, which led to the signing of millionaire contracts with governments and individuals. On July 5, 2015 the firms was hacked and about 400 GB of sensitive data had been published on WikiLeaks, showing a picture of clients and contacts of dubious morality. The previous year the company had already been criticized by The Citizen Lab which published a report\cite{ht_citizenlab} that connect Hacking Team with some actions of undemocratic governments. The stolen data not only confirmed the work conducted by The Citizen Lab, but designed a bigger picture. Hacking Team's customers included Italian and American intelligence agencies, but also private companies, arms dealers and governments such as Sudan (embargoed by the UN) and Egypt (full customer list on Wikipedia\cite{ht_wikipedia}). In addition, situations of favoritism emerged with unjustified collaboration between the private company and high officials of the government intelligence services\cite{attacco_ai_pirati}. The same Hacking Team's business can be found in other companies in the same industry, such as the Anglo-German Gamma Group (previously Gamma International), or the Israel based NICE Ltd. From these events emerge that the problems associated with this type of software do not concern only their use, but also their sale. Even if the European Commission in 2014 (EU regulation 1382/2014\cite{eu_1382/2014}) introduced intrusions software in the list of dual-use products, a list of goods, technology and software that can be used both for civil and military application, and are subject to more stringent export regulations, it's clearly necessary a more strict regulation of the providers company\cite{trojan_and_export} and, a way to identify the responsibilities of the involved players. The question is whether firms are the right type of entity to develop a such powerful technology. Developing intrusion software can be expensive for national agencies and private company can be a valid resource. What have to be kept in mind is that in a such sensitive field the social nature of the company must be well defined and constantly controlled by the government. As Filippo Pierozzi wrote in his report, Hacking Team had an hybrid nature, the firm had "a strong nationalistic connotation" \cite{ht_CCSIRS} but it did not hesitate to sell the product to possible enemy of the nation. In this sense Hacking Team was a business only firm, but has gained so much power that it can afford favors from very influential people. For these reasons, another question is who is responsible for the humans rights violation perpetrated with that software. Donaldson and Dunfee\cite{business_ethics} argue that firms have "free space" to select their moral value within the boundaries of certain "moral minima". Governments instead, if they call themselves liberals, have to respect their moral values and their citizens. In the Italian scandal emerged that not only Hacking Team company is morally responsible of crimes perpetrated by their software, but some guilt can be attributed to governments and commissions that have not properly supervised. In such sensitive partnership between public and private, that involves diplomatic relations and national security field, the government have to protect its interests and relationships with other states, imposing a complete transparency from the private company. Players that are not agree with the moral standards of a firm have to be free to don't do business with this company, in order to not be held responsible for any future violation of any kind\cite{sep-ethics-business}. One last discussion that I want to introduce concerns the possible implications that selling a software "ready-to-use", such as RCS Galileo, may have. This software, thanks to their simplicity and a short training course, allow anyone, without background in computer security field to perpetrate attack against other people. The difference between selling this software versus vulnerabilities only is in the audience size. While the sale of vulnerabilities shrinks the pool of buyers to experts only, "ready-to-use" software has a larger pool of buyers. Therefore the problems of this technology is not only related to their uncontrolled sale to government and subversive, but also to the fact that this software is sold to individuals or companies to spy on their opponents or employees, without any right or justification in doing this.          

\subsection*{Vulnerabilities like diamonds, rare and precious}
\addcontentsline{toc}{subsection}{Vulnerabilities like diamonds, rare and precious}
\paragraph{}
In this last paragraph I complete the speech concerning the issue of malware deployment by law enforcement, expanding the exploitation phase aspects. As already said, malware (and vulnerabilities) can be seen as a wasting resource and according to Paul Ohm this intrinsic characteristic make government abuse less likely\cite{malware_as_wasting_resource}, however the same characteristic can lead to other problems. An important element during a trial is the possibility for the defense to be able to fully analyze the work of the prosecution. Of course sometimes this is not possible for reasons of secrecy, but with the use of malware this can become the practice, moreover, hypothetically, thanks to the enormous amount of information that can be recovered with these tools, an entire process could be based only on the evidence gathered with spyware. What can happen is that due to the need not to burn a vulnerability, more information than the necessary are kept secret and some details of the investigation are omitted from the report, causing a lack of usable evidence in defense of the accused. Another issue can be identified thinking about the actions malware can perform once it is installed on a device: not only collect information but also interact with programs and files. This full access to the resources with the impossibility to make the source code of the intrusion software public for review, complicates being able to prove if any abuse, orchestration or falsification of the evidence took place during the investigation phases. Even if law enforcement and government agencies are trusted element of the state that have to act in the interest of citizens (while firms such as Hacking Team are not), these latter aspects pointed out, one more time, how easy is to abuse these tools and how important is to regulate and limit their use.        