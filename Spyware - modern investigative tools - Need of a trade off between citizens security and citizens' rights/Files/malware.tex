\paragraph{}
The term malware stand for malicious software, it is a generic term that includes Trojan horses and spyware, and describe any software with subversive purpose \cite{malware_definition}.
Trojans and spyware, after infecting a machine, aims to collect files, monitor user operations, turn on the webcam and capture pictures, and, at the end, reports the gathered information to a remote server controlled by the attacker. In some cases this software can be designed to remotely execute other programs on the victim machine. 
In this work I will often use the general term malware to refer to any kind of malicious software that act as Trojans.

\paragraph{}
Perhaps is already clear from the previous definition that malware is an extremely powerful tool in the digital age, both for illegal activities and for defense purposes. However, before discussing how this software can be used, some other considerations about their nature have to be pointed out, with the aim of better understanding which are the benefits and the risks that this kind of technology brings along with it. First I try to point out which are the properties that characterize a malware. Then I try to sum up which are the required steps to take advantage of these tools.

\paragraph{}
For our purpose I can summarize malware's characteristics as hidden, boundless, invasive and wasting resources. One of the key points that allows this kind of malicious software to be so powerful is because it works silently, the user ignore its presence, and it is extremely difficult to discover. Another strength is related to the fact that this technology is designed to work in the cyberspace where all the conventional boundaries have been destroyed. In this scenario anyone with a computer, and enough knowledge, can deploy an attack against someone located miles away from him. The invasiveness of a malware is due to the fact that the security countermeasure are bypassed and every file, program, microphone, webcam and so on can be accessed remotely and used without the user consent. Look Trojans as "wasting resource" \cite{malware_as_wasting_resource} is necessary to understand some problems that I will explore in the next chapters. Before being able to develop a malware that can take controls of a computer, a vulnerability in a program must be discovered and correctly exploited. A vulnerability is a weakness in a software caused by a not quite correct implementation functionalities. Programmers, software houses, researchers are constantly involved in looking up to this kind of weak point with the purpose of fix it. On the other side also hackers or other groups are interested in this field with the aim to sell vulnerabilities or use it. Having said that, and having pointed out that this is a match between who want to fix weakness in name of security against who want to take advantage of it, what remain to understand is that when a malware is discovered for the first time, also the exploited vulnerabilities  become of public domain and this mean that programmers start to fix them and the weaknesses are no longer available.

\paragraph{}
As Jonathan Mayer identified in his works, utilization of a malware can be divided in four distinct steps: "delivery, exploitation, execution and reporting" \cite{malware_steps}. Even if Mayer make this consideration explicitly regard to government malware, I think that this workflow suits well also to a more general context, where the attacker, the one who benefits of the use of this tool, can be a government, a hacker or anyone else. In the delivery phase the attacker have to find a way to install on the victim machine the malicious software. This can be done in different ways such as using social engineer techniques, inducing the target to click on a suspicious link, can be done manually, breaking into the office of the target, or can be done compromising an infrastructure in order to be able to monitor the entire data flow of the website. The exploitation phase require to tail the malicious software according to the target information, the data want to be collected and the available vulnerabilities. In the third step, the malware executes its behaviour and start collecting information or do whatever it is programmed to do. An important key point for some of the following discussions is that this phase not always has a deadline, certain malware can remain operational for a long period and neither the attacker stop it. And finally, the Trojan have to report the information to server that is under control of the attacker. 


